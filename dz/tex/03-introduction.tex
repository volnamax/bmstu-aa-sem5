\chapter*{ВВЕДЕНИЕ}
\addcontentsline{toc}{chapter}{ВВЕДЕНИЕ}


Цель данной лабораторной работы --- описать четырьмя графовыми моделями (графом управления, информационным графом, операционной историей, информационной историей) последовательный алгоритм либо фрагмент алгоритма, содержащий от 15 значащих строк кода и от двух циклов, один из которых является вложенным в другой.

Для достижения поставленной цели необходимо выполнить следующие задачи:
\begin{enumerate}
	\item описать алгоритм кластеризации к-средних;
	\item исследовать и разработать графовые модели для реализации алгоритма кластеризации k-средних;
	\item провести анализ графовых моделей и обосновать возможность \guillemotleft распараллеливания\guillemotright{} алгоритма кластеризации k-средних.
\end{enumerate}


\chapter{Выполнение задания}
В листинге~\ref{lst:k.cpp} приведена реализация алгоритма k-средних.


\includelistingpretty
{k.cpp} % Имя файла с расширением (файл должен быть расположен в директории inc/lst/)
{c++} % Язык программирования (необязательный аргумент)
{Реализация алгоритма k-средних} % Подпись листинга

\clearpage

\section{Графовые представления}

На рисунке~\ref{img:gu} представлен граф управления.
На рисунке~\ref{img:ig} представлен информационный граф.
На рисунке~\ref{img:oh} представлена операционная история. 
На рисунке~\ref{img:ih} представлена информационная история.

\newpage
\includeimage
{gu} % Имя файла без расширения (файл должен быть расположен в директории inc/img/)
{f} % Обтекание (без обтекания)
{h} % Положение рисунка (см. figure из пакета float)
{0.2\textwidth} % Ширина рисунка
{Граф управления} % Подпись рисунк
\clearpage
\newpage
\includeimage
{ig} % Имя файла без расширения (файл должен быть расположен в директории inc/img/)
{f} % Обтекание (без обтекания)
{h} % Положение рисунка (см. figure из пакета float)
{1\textwidth} % Ширина рисунка
{Информационный граф} % Подпись рисунк
\clearpage
\newpage
\includeimage
{oh} % Имя файла без расширения (файл должен быть расположен в директории inc/img/)
{f} % Обтекание (без обтекания)
{h} % Положение рисунка (см. figure из пакета float)
{1\textwidth} % Ширина рисунка
{Операционная история} % Подпись рисунк
\clearpage
\newpage
\includeimage
{ih} % Имя файла без расширения (файл должен быть расположен в директории inc/img/)
{f} % Обтекание (без обтекания)
{h} % Положение рисунка (см. figure из пакета float)
{1\textwidth} % Ширина рисунка
{Информационная история} % Подпись рисунк
\clearpage

\newpage
\section{Распараллеливание алгоритма}

Проанализировав графы~\ref{img:gu}~--~\ref{img:ih} можно сделать вывод, что распараллеливанию поддаются следующем этапы алгоритма:

\begin{itemize}
	\item инициализация центроидов;
	\item назначение точек кластерам;
	\item обновление центроидов кластеров.
\end{itemize}

\section*{Вывод}

В данном разделе была разработана реализация алгоритма k-средних, разработаны графовые модели, а также описаны возможные варианты распараллеливания алгоритма.

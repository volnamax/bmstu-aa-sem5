\chapter*{ВВЕДЕНИЕ}
\addcontentsline{toc}{chapter}{ВВЕДЕНИЕ}

Матрицей --- называется прямоугольная таблица, представляющая собой массив элементов, для которой определены математические операции. 
Элементами матрицы могут служить числа, алгебраические символы или математические функции~\cite{matrix}. 
Умножение матриц широко применяется в различных задачах, и поэтому изучение алгоритмов для его выполнения является важным вопросом на сегодняшний день~\cite{matrix_in}.

Цель данной лабораторной работы --- исследование следующих алгоритмов умножения матриц:

\begin{itemize}
	\item классического алгоритма;
	\item алгоритма Штрассена;
	\item алгоритма Винограда;
	\item оптимизированная версия алгоритма Винограда в соответствии с заданным вариантом, а именно: 
		\begin{itemize}
			\item заменить умножение на 2 на побитовый сдвиг;
			\item заменить операцию $x = x + k$ на $x += k$;
			\item вычислять заранее некоторые слагаемые для алгоритма.
		\end{itemize}
\end{itemize}


Для достижения поставленной цели необходимо выполнить следующие задачи:
\begin{itemize}
	\item разработать требуемые алгоритмы;
	\item оценить трудоемкость рассматриваемых алгоритмов;
	\item провести сравнительный анализ времени выполнения реализуемых алгоритмов и занимаемой памяти.
\end{itemize}

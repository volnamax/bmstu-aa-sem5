\chapter{Конструкторский раздел}

В этом разделе представлены: требования к программному обеспечению и схемы реализуемых алгоритмов.


\section{Требования к программному обеспечению}

На вход принимаются две матрицы c размерами.
На выходе программе требуется получить результирующую матрицу после умножения входных матриц.

\section{Описание используемых типов данных}
При реализации алгоритмов будет использован следующий тип данных, матрица --- двумерный массив значений целочисленного типа.

\section{Разработка алгоритмов}

Алгоритмы принимают на вход матрицы $mtrx$ и $matry$.
На выходе у алгоритмов результат перемножения двух матриц, записанный в матрицу $matr\_res$.
На рисунках~\ref{img:classic}~---~~\ref{img:strassen} приведены схемы для реализуемых алгоритмов.
	
\includeimage
	{classic} % Имя файла без расширения (файл должен быть расположен в директории inc/img/)
	{f} % Обтекание (без обтекания)
	{h} % Положение рисунка (см. figure из пакета float)
	{1\textwidth} % Ширина рисунка
	{Схема классического алгоритма умножения метриц} % Подпись рисунк
	\clearpage

\includeimage
	{vinograde} % Имя файла без расширения (файл должен быть расположен в директории inc/img/)
	{f} % Обтекание (без обтекания)
	{h} % Положение рисунка (см. figure из пакета float)
	{0.7\textwidth} % Ширина рисунка
	{Схема aлгоритма Винограда} % Подпись рисунк
	\clearpage

\includeimage
	{vino_oprimo} % Имя файла без расширения (файл должен быть расположен в директории inc/img/)
	{f} % Обтекание (без обтекания)
	{h} % Положение рисунка (см. figure из пакета float)
	{0.7\textwidth} % Ширина рисунка
	{Схема оптимизированного aлгоритма Винограда} % Подпись рисунк
	\clearpage

\includeimage
	{strassen} % Имя файла без расширения (файл должен быть расположен в директории inc/img/)
	{f} % Обтекание (без обтекания)
	{h} % Положение рисунка (см. figure из пакета float)
	{1\textwidth} % Ширина рисунка
	{Схема алгоритма Штрассена} % Подпись рисунк
\clearpage

\section{Модель вычислений}
Для последующего вычисления трудоемкости необходимо ввести модель вычислений:

\begin{enumerate}
	\item операции из списка (\ref{eq:opers}) имеют трудоемкость 1;
	\begin{equation}
		\label{eq:opers}
		+, -, *, /, \%, ==, !=, <, >, <=, >=, [], ++, {-}-
	\end{equation}
	\item трудоемкость оператора выбора \textit{if условие then A else B} рассчитывается как (\ref{eq:if});
	\begin{equation}
		\label{eq:if}
		f_{if} = f_{\text{условия}} +
		\begin{cases}
			f_A, & \text{если условие выполняется,}\\
			f_B, & \text{иначе.}
		\end{cases}
	\end{equation}
	\item трудоемкость цикла рассчитывается как (\ref{eq:for});
	\begin{equation}
		\label{eq:for}
		f_{for} = f_{\text{инициализации}} + f_{\text{сравнения}} + N(f_{\text{тела}} + f_{\text{инкремента}} + f_{\text{сравнения}})
	\end{equation}
	\item трудоемкость вызова функции равна 0.
\end{enumerate}


\section{Трудоемкость алгоритмов}

Трудоемкость инициализации результирующей матрицы учитываться не будет, так как данное действие есть во всех алгоритмах и не является трудоемким.

Обозначения для расчета трудоемкости :
\begin{itemize}
	\item N -- кол-во строк первой матрицы;
	\item M -- кол-во столбцов первой матрицы и кол-во строк второй матрицы;
	\item Q -- кол-во столбцов второй матрицы.
	
\end{itemize}

\textbf{Трудоемкость классический алгоритм перемножения матриц}

Трудоемкость стандартного алгоритма умножения матриц состоит из:

\begin{itemize}
	\item внешнего цикла по $i \in [1..M]$, трудоемкость которого: $f = 2 + M \cdot (2 + f_{body})$;
	\item цикла по $j \in [1..N]$, трудоемкость которого: $f = 2 + N \cdot (2 + f_{body})$;
	\item цикла по $k \in [1..Q]$, трудоемкость которого: $f = 2 + 14 \cdot Q$.
\end{itemize}

Поскольку трудоемкость стандартного алгоритма равна трудоемкости внешнего цикла, то
\begin{equation}
	\label{for:classic}
	f_{classic} = 2 + M \cdot (4 + N \cdot (4 + 14Q)) = 2 + 4M + 4MN + 14MQN \approx 14MQN
\end{equation}

\textbf{Трудоемкость алгоритм Винограда}

Трудоемкость алгоритма Винограда складывается из:
\begin{itemize}
	\item трудоемкости создания и инициализации массивов $row$ и $col$:
	\begin{equation}
		\label{eqn:f-arrs}
		f_{arrs} = f_{row} + f_{col}
	\end{equation}
	\item трудоемкость заполнения массива $row$:
	\begin{equation}
		\begin{gathered}
			f_{row} = 2 + N \cdot (2 + 4 + \frac{M}{2} \cdot (4 + 6 + 1 + 2 + 3 \cdot 2)) = \\
			2 + 6M + \frac{19MN}{2};
		\end{gathered}
	\end{equation}
	\item трудоемкость заполнения массива $col$:
	\begin{equation}
		\begin{gathered}
			f_{col} = 2 + Q \cdot (2 + 4 + \frac{M}{2} \cdot (4 + 6 + 1 + 2 + 3 \cdot 2)) = \\
			2 + 6Q + \frac{19MQ}{2};
		\end{gathered}
	\end{equation}
	\item трудоемкость цикла умножения для четных размеров:
	\begin{equation}
		\begin{gathered}
			f_{mul} = 2 + N \cdot (4 + Q \cdot (13 + 32\frac{M}{2})) = 2 + 4N + \\
			+ 13NQ + \frac{32MNQ}{2} = 2 + 4N + 13NQ + 16MNQ;
		\end{gathered}
	\end{equation}
	\item трудоемкость цикла, выполняемого в случае нечетных размеров матрицы:
	\begin{equation}
		f_{oddLoop} = 3 + 
		\begin{cases}
			0, & \text{четная,} \\
			2 + N \cdot (4 + Q \cdot (2 + 14)), & \text{иначе.}
		\end{cases}
	\end{equation}
\end{itemize}
Таким образом, для нечетного размера матрицы имеем:
\begin{multline}
	f_{odd} = f_{arrs} + f_{row} + f_{col} + f_{mul} + f_{oddLoop}  \approx 16MNQ;
\end{multline}
для четного:
\begin{multline}
	f_{even} = f_{arrs} + f_{row} + f_{col} + f_{mul} + f_{oddLoop} \approx 16MNQ.
\end{multline}

\textbf{Оптимизированный алгоритм Винограда}

Трудоемкость оптимизации алгоритма Винограда осуществляется следующим образом:
\begin{itemize}
	\item операция $x = x + k$ заменяется на операцию $x += k$;
	\item операция $x \cdot 2$ заменяется на $x << 1$;
	\item некоторые значения для алгоритма вычисляются заранее.
\end{itemize}

Тогда трудоемкость алгоритма Винограда c примененными оптимизациями складывается из:
\begin{itemize}
	\item трудоемкости предвычисления значения $\frac{M}{2}$, равной 3;
	\item трудоемкости $f_{arrs}$ (\ref{eqn:f-arrs}) создания и инициализации массивов $row$ и $col$;
	\item трудоемкость заполнения массива $row$:
	\begin{equation}
		\begin{gathered}
			f_{row} = 2 + N \cdot (2 + 2 + \frac{M}{2} \cdot (2 + 2 + 2 + 7)) = \\
			2 + 2N + \frac{13MN}{2};
		\end{gathered}
	\end{equation}
	\item трудоемкость заполнения массива $col$:
	\begin{equation}
		\begin{gathered}
			f_{col} = 2 + Q \cdot (2 + 2 + \frac{M}{2} \cdot (2 + 2 + 2 + 7)) = \\
			2 + 2Q + \frac{13MQ}{2};
		\end{gathered}
	\end{equation}
	\item трудоемкость цикла умножения для четных размеров:
	\begin{equation}
		\begin{gathered}
			f_{mul} = 2 + N \cdot (4 + Q \cdot (4 + 2 + \frac{M}{2} \cdot (2 + 2 + 3 + 6 + 2 + 6))) = \\
			= 2 + 4N + 13NQ + \frac{32MNQ}{2} = 2 + 4N + 13NQ + 16MNQ;
		\end{gathered}
	\end{equation}
	\item трудоемкость цикла, выполняемого в случае нечетных размеров матрицы:
	\begin{equation}
		f_{oddLoop} = 3 + 
		\begin{cases}
			0, & \text{четная,} \\
			2 + N \cdot (4 + Q \cdot (2 + 11)), & \text{иначе.}
		\end{cases}
	\end{equation}
\end{itemize}
Таким образом, для нечетного размера матрицы имеем:
\begin{multline}
	f_{odd} = f_{arrs} + f_{row} + f_{col} + f_{mul} + f_{oddLoop}  \approx \frac{19MNQ}{2};
\end{multline}
для четного:
\begin{multline}
	f_{even} = f_{arrs} + f_{row} + f_{col} + f_{mul} + f_{oddLoop}  \approx \frac{19MNQ}{2}.
\end{multline}


\textbf{Трудоемкость алгоритма Штрассена}

Трудоемкость алгоритма Штрассена осуществляется следующим образом:

Трудоемкость сложения или вычитания двух матриц размера $N \times N$, рассчитывается по формуле \eqref{eq:matrix_sum}.
\begin{equation}
	\label{eq:matrix_sum}
	f_{sum}(N) = 2 + N \cdot (2 + 2 + N \cdot (2 + 8)) = 10N^2 + 4N + 2
\end{equation}

Тогда трудоемкость описывается рекуррентной формулой \eqref{eq:strassen_form}.

\begin{equation}
	\label{eq:strassen_form}
	T(n) = 
	\begin{cases}
		7, & n = 1\\
		7T(\frac{n}{2}) + 18f_{sum}(\frac{n}{2}) + 65, & n > 1
	\end{cases}
\end{equation}

Тогда по формуле \eqref{eq:strassen_comp} возможно рассчитать трудоемкость для матриц порядка $n$.
\begin{equation}
	\label{eq:strassen_comp}
	\begin{gathered}
		T(n) = 7^{\log_{2}{n}} T(1) + \sum_{i=0}^{(\log_{2}{n}) - 1} (7^i (18f_{sum}(\frac{n}{2 ^ {i + 1}}) + 65))
	\end{gathered}
\end{equation}

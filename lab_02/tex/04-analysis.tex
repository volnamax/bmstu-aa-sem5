\chapter{Аналитический раздел}

\textit{Классический алгоритм} умножения матриц является базовым методом, в котором каждый элемент результирующей матрицы вычисляется с использованием тройного вложенного цикла. 
Данный алгоритм --- это реализация математического определения умножения матриц.
Его асимптотическая сложность равна $O(n^3)$, такая ассимптотика делает его менее эффективным для больших матриц~\cite{matrix}.

\textit{Алгоритм Винограда} вводит предварительную обработку матриц для оптимизации вычислений. Он использует дополнительные массивы для хранения промежуточных результатов и требует меньше умножений по сравнению с классическим методом. 
Это делает реализацию этого алгоритма более эффективной по времени выполнению. 
Его асимптотическая сложность $O(n^{2.3755})$~\cite{vino}.

\textit{Алгоритм Штрассена} представляет собой рекурсивный метод, который основан на разделении матриц на подматрицы и использовании 7 умножений вместо 8, для матриц размерностью $2\times2$. Его асимптотическая сложность $O(n \log_2 7)$, делает его привлекательным для работы с большими матрицами~\cite{shtrassen}. 

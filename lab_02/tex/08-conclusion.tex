\chapter*{ЗАКЛЮЧЕНИЕ}
\addcontentsline{toc}{chapter}{ЗАКЛЮЧЕНИЕ}


Цель работы достигнута, проведен сравнительный анализ следующих алгоритмов умножения матриц:
\begin{itemize}
	\item классического алгоритма;
	\item алгоритма Штрассена;
	\item алгоритма Винограда;
	\item оптимизированная версия алгоритма Винограда;
\end{itemize}

В ходе выполнения лабораторной работы были решены все задачи:
\begin{itemize}
	\item разработаны требуемые алгоритмы;
	\item оценена трудоемкость рассматриваемых алгоритмов;
	\item проведен сравнительный анализ времени выполнения реализуемых алгоритмов и занимаемой памяти.
	
	В результате исследования реализуемых алгоритмов по времени выполнения были сделаны следующие выводы:
	\begin{enumerate}
		\item реализация оптимизированного алгоритма Винограда оказалась наиболее эффективной по времени, независимо от размерности входных матриц (см. рисунки~\ref{img:oneToHundred}~--~\ref{img:twoToHundred}).
		
		\item реализация алгоритма Штрассена по итогам исследования оказалась наименее эффективной по времени (см. рисунок~\ref{img:twoToHundred}).
		Полученный результат объясняется малыми размерами матриц. Однако из-за того, что алгоритм рекурсивный, достижение необходимых размеров матриц, при которых реализация алгоритма Штрассена показала бы преимущество, оказывается невозможным.
		
		\item на матрицах нечетного размера оптимизированная и неоптимизированная реализация алгоритмов Винограда работают медленнее, чем на матрицах четного размера (см. рисунки~\ref{img:oneToHundred}~--~\ref{img:twoToHundred}). Это происходит из-за дополнительных вычислений для крайних строк и столбцов в результирующей матрице.
	\end{enumerate}
	
	В результате теоретической оценки объема используемой памяти для реализаций 
	алгоритмов были сделаны следующие выводы:
	\begin{enumerate}
		\item реализация классического алгоритма умножения матриц требует наименьших расходов по памяти.
		\item реализация алгоритма Штрассена, напротив, является самой требовательной по памяти за счет использования вспомогательных подматриц для выполнения расчетов и рекурсивных вызовов.
		\item оптимизированная реализация алгоритма Винограда более ресурсозатратна по сравнению с неоптимизированной, поскольку включает в себя использование дополнительных переменных для хранения промежуточных расчетов.
	\end{enumerate}
\end{itemize}
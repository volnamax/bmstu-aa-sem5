\chapter{Исследовательская часть}

В данном разделе будут приведены: технические характеристики устройства, демонстрация работы программы, сравнительный анализ времени выполнения реализуемых алгоритмов и занимаемой памяти.

\section{Технические характеристики}

Технические характеристики устройства, на котором выполнялись замеры по времени, представлены далее.

\begin{itemize}
	\item Процессор: Ryzen 5 4600H, тактовая частота ЦПУ 3.0 ГГц, максимальная частота процессора 4.0 ГГц~\cite{ryzen}.
	\item Оперативная память: 16 ГБайт.
	\item Операционная система: Windows 10 Pro 64-разрядная система \cite{windows}.
\end{itemize}

При замерах времени ноутбук был включен в сеть электропитания и был нагружен только системными приложениями.

\section{Демонстрация работы программы}

На рисунке~\ref{img:demka} представлено визуальное представление работы разработанного программного обеспечения. В частности, продемонстрированы результаты сортировки для всех реализаций алгоритмов, при входном массиве $[15, 7, 9, 30, 8, 9, 9, 10, 45]$.

\includeimage
	{demka} % Имя файла без расширения (файл должен быть расположен в директории inc/img/)
	{f} % Обтекание (без обтекания)
	{h} % Положение рисунка (см. figure из пакета float)
	{0.5\textwidth} % Ширина рисунка
	{Демонстрация работы программы} % Подпись рисунк

\section{Время выполнения реализаций алгоритмов}

Результаты замеров времени выполнения реализаций алгоритмов сортировок приведены в таблицах \ref{tbl:time_measurements} -- \ref{tbl:time_measurements_rand}.
Замеры времени проводились на массивах одного размера и усреднялись для каждого набора одинаковых экспериментов.

В таблицах \ref{tbl:time_measurements} -- \ref{tbl:time_measurements_rand} используются следующие обозначения: 
\begin{itemize}
	\item Блочная --- реализация алгоритма блочной сортировки;
	\item Бусинами --- реализация алгоритма сортировки бусинами;
	\item Выбором --- реализация алгоритма сортировки выбором.
\end{itemize}



\clearpage
\begin{table}[H]
	\begin{center}
		\begin{threeparttable}
			\captionsetup{justification=raggedright,singlelinecheck=off}
			\caption{Время работы реализации алгоритмов на неотсортированных массивах (в мс)}
			\label{tbl:time_measurements}
			\begin{tabular}{|c|c|c|c|}
				\hline
				Размер массива & Блочная & Бусинами & Выбором \\
			
			\hline
			100 & 0.063 & 6.453 & 7.547 \\
			\hline
			200 & 0.172 & 14.469 & 13.969 \\
			\hline
			300 & 0.328 & 22.875 & 22.547 \\
			\hline
			400 & 0.531 & 30.734 & 31.562 \\
			\hline
			500 & 0.781 & 39.281 & 39.641 \\
			\hline
			600 & 1.094 & 51.594 & 50.156 \\
			\hline
			700 & 1.406 & 60.031 & 59.406 \\
			\hline
			800 & 1.797 & 71.359 & 69.984 \\
			\hline
			900 & 2.250 & 79.203 & 80.453 \\
			\hline
			1000 & 2.719 & 88.953 & 88.203 \\
			
				\hline
			\end{tabular}
		\end{threeparttable}
	\end{center}
\end{table}

\begin{table}[H]
	\begin{center}
		\begin{threeparttable}
			\captionsetup{justification=raggedright,singlelinecheck=off}
			\caption{Время работы реализации алгоритмов на отсортированных в обратном порядке массивах (в мс)}
			\label{tbl:time_measurements_sorted}
			\begin{tabular}{|c|c|c|c|}
				\hline
				Размер массива & Блочная & Бусинами & Выбором \\
				\hline
				100 & 0.063 & 7.250 & 7.094 \\
				\hline
				200 & 0.172 & 14.062 & 14.578 \\
				\hline
				300 & 0.328 & 22.313 & 22.375 \\
				\hline
				400 & 0.516 & 31.656 & 30.828 \\
				\hline
				500 & 0.781 & 39.031 & 40.203 \\
				\hline
				600 & 1.078 & 46.188 & 49.141 \\
				\hline
				700 & 1.422 & 59.031 & 58.266 \\
				\hline
				800 & 1.797 & 69.828 & 71.109 \\
				\hline
				900 & 2.234 & 79.031 & 79.344 \\
				\hline
				1000 & 2.703 & 88.406 & 89.062 \\
				\hline
			\end{tabular}
		\end{threeparttable}
	\end{center}
\end{table}

\begin{table}[H]
	\begin{center}
		\begin{threeparttable}
			\captionsetup{justification=raggedright,singlelinecheck=off}
			\caption{Время работы реализации алгоритмов на отсортированных массивах (в мс)}
			\label{tbl:time_measurements_rand}
			\begin{tabular}{|c|c|c|c|}
				\hline
				Размер массива & Блочная & Бусинами & Выбором \\
				\hline
				100 & 0.062500 & 7.250000 & 7.093750 \\
				\hline
				200 & 0.171875 & 14.062500 & 14.578125 \\
				\hline
				300 & 0.328125 & 22.312500 & 22.375000 \\
				\hline
				400 & 0.515625 & 31.656250 & 30.828125 \\
				\hline
				500 & 0.781250 & 39.031250 & 40.203125 \\
				\hline
				600 & 1.078125 & 46.187500 & 49.140625 \\
				\hline
				700 & 1.421875 & 59.031250 & 58.265625 \\
				\hline
				800 & 1.796875 & 69.828125 & 71.109375 \\
				\hline
				900 & 2.234375 & 79.031250 & 79.343750 \\
				\hline
				1000 & 2.703125 & 88.406250 & 89.062500 \\
				\hline
			\end{tabular}
		\end{threeparttable}
	\end{center}
\end{table}




На рисунках \ref{img:noSort} -- \ref{img:sort} изображены графики зависимостей времени выполнения реализаций сортировок от размеров массивов.



\includeimage
{noSort} % Имя файла без расширения (файл должен быть расположен в директории inc/img/)
{f} % Обтекание (без обтекания)
{h} % Положение рисунка (см. figure из пакета float)
{1\textwidth} % Ширина рисунка
{Сравнение реализаций алгоритмов по времени выполнения на неотсортированных массивах} % Подпись рисунк


\includeimage
{reverseSort} % Имя файла без расширения (файл должен быть расположен в директории inc/img/)
{f} % Обтекание (без обтекания)
{h} % Положение рисунка (см. figure из пакета float)
{1\textwidth} % Ширина рисунка
{Сравнение реализаций алгоритмов по времени выполнения на отсортированных в обратном порядке массивах} % Подпись рисунк




\includeimage
{sort} % Имя файла без расширения (файл должен быть расположен в директории inc/img/)
{f} % Обтекание (без обтекания)
{h} % Положение рисунка (см. figure из пакета float)
{1\textwidth} % Ширина рисунка
{Сравнение реализаций алгоритмов по времени выполнения на отсортированных массивах} % Подпись рисунк
\clearpage

\section{Вывод}

В результате исследования реализуемых алгоритмов по времени выполнения можно сделать следующие выводы:
\begin{enumerate}
	\item Алгоритм блочной сортировки демонстрирует значительно лучшую производительность на всех типах массивов, особенно выделяясь на неупорядоченных массивах (см. рисунки~\ref{img:noSort}~--~\ref{img:sort}). его реализация значительно превосходит другие алгоритмы, что делает его предпочтительным выбором для разнообразных данных.
	\item Алгоритм сортировки "Бусинами" показывает хорошие результаты на упорядоченных в обратном порядке и неупорядоченных массивах, но его  производительность снижается на массивах, упорядоченных по возрастанию (см. рисунки~\ref{img:noSort}~--~\ref{img:sort}). Это делает его подходящим для определенных случаев использования, особенно когда данные не предварительно отсортированы.
	\item Реализация алгоритма сортировки выбором  оказалась менее эффективной в сравнении с другими алгоритмами на всех типах массивов (см. рисунки~\ref{img:noSort}~--~\ref{img:sort}).
\end{enumerate}



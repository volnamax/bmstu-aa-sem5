\chapter{Конструкторский раздел}

В этом разделе представлены: требования к программному обеспечению, описание используемых типов данных, оценка трудоемкости и схемы реализуемых алгоритмов.


\section{Требования к программному обеспечению}


Программа должна поддерживать два режима работы: режим массового замера времени и режим сортировки введенного массива.

Режим массового замера времени должен обладать следующей функциональностью:
\begin{itemize}
	\item генерировать массивы различного размер для проведения замеров;
	\item осуществлять массовый замер, используя сгенерированные данные;
	\item результаты массового замера должны быть представлены в виде таблицы и графика.
\end{itemize}

К режиму сортировки выдвигается следующий ряд требований:
\begin{itemize}
	\item возможность работать с массивами разного размера, которые вводит пользователь;
	\item наличие интерфейса для выбора действий;
	\item на выходе программы массив, отсортированный тремя алгоритмами по возрастанию.
\end{itemize}

\section{Описание используемых типов данных}

При реализации алгоритмов будут использованы следующие структуры и типы данных:
\begin{itemize}
	\item целое число представляет количество элементов в массиве;
	\item список целых чисел;
\end{itemize}

\section{Разработка алгоритмов}

Алгоритмы принимают на вход массив $arr$.
На выходе у алгоритмов отсортированный массив $arr$.
На рисунках~\ref{img:block_sort}~---~~\ref{img:selection} приведены схемы для реализуемых алгоритмов.

\includeimage
	{block_sort} % Имя файла без расширения (файл должен быть расположен в директории inc/img/)
	{f} % Обтекание (без обтекания)
	{h} % Положение рисунка (см. figure из пакета float)
	{1\textwidth} % Ширина рисунка
	{Схема алгоритма блочной сортировки} % Подпись рисунк
 \clearpage

\includeimage
	{block_merge} % Имя файла без расширения (файл должен быть расположен в директории inc/img/)
	{f} % Обтекание (без обтекания)
	{h} % Положение рисунка (см. figure из пакета float)
	{1\textwidth} % Ширина рисунка
	{Схема вспомогательной функции merge для алгоритма блочной сортировки} % Подпись рисунк
\clearpage

\includeimage
	{breads} % Имя файла без расширения (файл должен быть расположен в директории inc/img/)
	{f} % Обтекание (без обтекания)
	{h} % Положение рисунка (см. figure из пакета float)
	{1\textwidth} % Ширина рисунка
	{Схема алгоритма сортировки бусинами} % Подпись рисунк
\clearpage

\includeimage
	{selection} % Имя файла без расширения (файл должен быть расположен в директории inc/img/)
	{f} % Обтекание (без обтекания)
	{h} % Положение рисунка (см. figure из пакета float)
	{0.4\textwidth} % Ширина рисунка
	{Схема алгоритма сортировки выбором} % Подпись рисунк
\clearpage



\section{Оценка трудоемкости алгоритмов}

Модель для оценки трудоемкости алгоритмов состоит из шести пунктов:
\begin{enumerate}
	\item $+, -, =, +=, -=, ==, ||, \&\&, <, >, <=, >=, <<, >>, []$ --- считается, что эти операции обладают трудоемкостью в 1 единицу;
	\item $*, /, *=, /=, \% $ --- считается, что эти операции обладают трудоемкостью в 2 единицы;
	\item трудоемкость условного перехода принимается за $0$;
	\item трудоемкость условного оператора рассчитывается по формуле \eqref{eq:if},
	\begin{equation}
		\label{eq:if}
		f_{if} = f_{\text{условия}} + 
		\begin{cases}
			min(f_1, f_2), & \text{лучший случай}\\
			max(f_1, f_2), & \text{худший случай}
		\end{cases},
	\end{equation}
	где $f_1$ --- трудоемкость блока, который вычисляется при выполнении условия, а $f_2$ --- трудоемкость блока, который вычисляется при невыполнении условия;
	\item трудоемкость цикла рассчитывается по формуле \eqref{eq:for},
	\begin{equation}
		\label{eq:for}
		\begin{gathered}
			f_{for} = f_{\text{инициализация}} + f_{\text{сравнения}} + M_{\text{итераций}} \cdot (f_{\text{тело}} +\\
			+ f_{\text{инкремент}} + f_{\text{сравнения}});
		\end{gathered}
	\end{equation}
	\item вызов подпрограмм и передача параметров принимается за $0$.
\end{enumerate}

\subsection{Трудоемкость алгоритма блочной сортировки}

В данной реализации размер блока обозначается как $k$, трудоемкость операции добавления и удаления элемента из вектора равна 2.


\textbf{Лучший случай:} массив отсортирован, элементы распределены равномерно (все блоки содержат одинаковое число элементов), расчет трудоемкости данного случая приведен в (\ref{сomplexity:block_best}).


\begin{equation}
	\label{сomplexity:block_best}
	\begin{gathered}
		f_{best} = 1 +1 + \frac{n}{k} \cdot(1 + 2+f_{shaker} + 2 + 1 + 4 + \\
		+ k \cdot (3 + 1 + 4)) + 1 + 1 + \\
		+ \frac{n}{k} \cdot (1 + 4 + 1 + 1 + 5 + 1 + 4 + 1 + 1 + n \cdot (5)) = \\
		= 4 + \frac{29\cdot n + n \cdot f_{shaker} + 5 \cdot n^2}{k}  + 8 \cdot n  = \\
		= 4 + 8 \cdot n + 29 \cdot \frac{n}{k} + n \cdot (14.5 + \frac{k}{2}) + \frac{5 \cdot n^2}{k}.
	\end{gathered}
\end{equation}

\textbf{Худший случай:} большое количество пустых блоков, массив отсортирован в обратном порядке (худший случай сортировки перемешиванием, которая используется в блочной сортировке), расчет трудоемкости приведен в выражении (\ref{сomplexity:block_worst}).

\begin{equation}
	\label{сomplexity:block_worst}
	\begin{gathered}
		f_{worst} = 1 +1 + \frac{n}{k} \cdot(1 + 2+f_{shaker} + 2 + 1 + 4 + \\
		+ k \cdot (3 + 1 + 4)) + 1 + 1 + \\
		+ \frac{n}{k} \cdot (1 + 4 + 1 + 1 + 5 + 1 + 4 + 1 + 1 + k \cdot (6)) = \\
		= 4 + \frac{29\cdot n + n \cdot f_{shaker} + 6 \cdot n^2}{k}  + 8 \cdot n  = \\
		= 4 + 8 \cdot n + 29 \cdot \frac{n}{k} + n \cdot (19.5 + \frac{k}{2}) + \frac{6 \cdot n^2}{k}.
	\end{gathered}
\end{equation}


\textbf{Вывод о трудоемкости блочной сортировки}

Данная реализация зависит от выбранного размера блока, а также от количества сортируемых элементов. В соответствии с выражениями (\ref{сomplexity:block_worst}, \ref{сomplexity:block_best})  операнды $\frac{n \cdot k}{2}$, $\frac{n^2}{k}$ и $\frac{n}{k}$ зависят от значений $n$ и $k$, первый приведенный операнд возрастает при увеличении $k$, другие убывают.


\subsection{Трудоемкость алгоритма сортировки выбором}

Трудоемкость сортировки выбором в худшем случае $O(N^2)$

Трудоемкость сортировки выбором в лучшем случае $O(N^2)$



\subsection{Трудоемкость алгоритма сортировки бусинами}

Трудоемкость в лучшем случае при отсортированном массиве и худшем случае при неотсортированном массиве в обратном порядке. 
Выведена формуле~(\ref{сomplexity:bead}).
\begin{equation}
	\label{сomplexity:bead}
	\begin{gathered}
		f_{best} = 3 + 4 + (N - 1) \cdot (2 + 2 + 
		\begin{cases}
			0, \\
			2,
		\end{cases}) 
		+ 1 + 2 + \\
		+ N \cdot (2 + 3 + L \cdot (3 + 5)) + 2 + M \cdot (2 + 3 + N \cdot (2 + 5 + 5) + \\
		+3 + (N - L) \cdot (2 + 5)) + 2 + N \cdot (2 + 8 + 8M) = \\
		= S + 19NM + 11N + 8M + 18 = O(S)
	\end{gathered}
\end{equation}

В формуле~(\ref{сomplexity:bead}) значения $S$ или $NL$ --- сумма всех элементов в изначальном массиве, $L$ --- значение каждого элемента в массиве,  M --- максимальное значение из элементов в массиве.

Исходя из результата формулы (\ref{сomplexity:bead}) можно понять, что лучшим случаем для сортировки будет случай если все значения массива будут соответствовать минимальному значению, а худшим случаем если значение массива будут большие значения. 


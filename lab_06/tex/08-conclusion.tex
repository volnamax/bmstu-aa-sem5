\chapter*{Заключение}
\addcontentsline{toc}{chapter}{Заключение}

Поставленная цель достигнута: была выполнена параметризация метода решения задачи коммивояжера на основе муравьиного алгоритма.

В ходе выполнения лабораторной работы были решены все задачи:

\begin{enumerate}
	\item описана задача коммивояжера;
	\item описаны методы решения задачи коммивояжера --- метод полного перебора и метод на основе муравьиного алгоритма;
	\item приведены схемы муравьиного алгоритма и алгоритма, позволяющего решить задачу коммивояжера методом полного перебора;
	\item разработан и реализован программный продукт, позволяющий решить задачу коммивояжера исследуемыми методами;
	\item сравнены по времени метод полного перебора и метод на основе муравьиного алгоритма;
	\item описаны и обоснованы полученные результаты в отчете о выполненной лабораторной работе.
\end{enumerate}

Исходя из полученных результатов, использование муравьиного алгоритма наиболее эффективно по времени при больших размерах матриц.
Так при размере матрицы, равном 2, муравьиный алгоритм меленее алгоритма полного перебора в 143 раза, а при размере матрицы, равном 9, муравьиный алгоритм быстрее алгоритма полного перебора в 11 раз, а при размере в 10 -- уже в 15 раз.
Следовательно, при размерах матриц больше 8 следует использовать муравьиный алгоритм, но стоит учитывать, что он не гарантирует оптимального решения, в отличие от метода полного перебора.
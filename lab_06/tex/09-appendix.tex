\chapter*{Приложение А}
\addcontentsline{toc}{chapter}{Приложение А}

\begin{center}
	\captionsetup{justification=raggedright,singlelinecheck=off}
	\begin{longtable}[c]{|l|l|l|l|}
		\caption{Параметризация для класса данных 1\label{tbl:param_kd1-1}, Elites --- количество элитных муравьев, Mistake --- ошибочность полученного результата}\\ \hline
		$\alpha$ & $\rho$ & Elites & Mistake
		\csvreader{inc/csv/class1.csv}{}
		{\\ \hline \csvcoli & \csvcolii & \csvcoliii & \csvcoliv}
		\\ \hline
	\end{longtable}
\end{center}


\chapter*{Приложение Б}
\addcontentsline{toc}{chapter}{Приложение Б}



\begin{center}
	\captionsetup{justification=raggedright,singlelinecheck=off}
	\begin{longtable}[c]{|l|l|l|l|}
		\caption{Параметризация для класса данных 2\label{tbl:param_kd1}, Elites --- количество элитных муравьев, Mistake --- ошибочность полученного результата}\\ \hline
		$\alpha$ & $\rho$ & Elites & Mistake
		\csvreader{inc/csv/class2.csv}{}
		{\\ \hline \csvcoli & \csvcolii & \csvcoliii & \csvcoliv}
		\\ \hline
	\end{longtable}
\end{center}


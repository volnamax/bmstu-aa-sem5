\chapter{Аналитическая часть}
В этом разделе будет представлена информация о задаче коммивояжера,
а также о способах её решения — методе полного перебора и методе на основе муравьиного алгоритма.

Задача коммивояжера \text{(англ. \textit{traveling salesman problem})} --- (задача о бродячем торговце) одна из самых важных задач всей транспортной логистики, в которой рассматриваются вершины графа, а также матрица смежности (для расстояния между вершинами)\,\cite{task}.
Задача заключается в том, чтобы найти такой порядок посещения вершин графа, при котором путь будет минимален, каждая вершина будет посещена лишь один раз, а возврат произойдет в начальную вершину.

Полный перебор для задачи коммивояжера имеет высокую сложность алгоритма ($n!$), где $n$ --- количество портов~\cite{full-comb}.
Суть в полном переборе всех возможных путей в графе и выбор наименьшего из них.
Решение будет получено, но имеются большие затраты по времени выполнения при уже небольшом количестве вершин в графе.

\section{Метод на основе муравьиного алгоритма}

\textbf{Муравьиный алгоритм} (англ. \textit{ant colony optimization}) --- метод решения задачи оптимизации, основанный на принципе поведения колонии муравьев~\cite{full-comb}.

Муравьи действуют, руководствуясь органами чувств.
Каждый муравей оставляет на своем пути феромоны, чтобы другие могли по ним ориентироваться.
При большом количестве муравьев наибольшее количество феромона остается на наиболее посещаемом пути, посещаемость же связана с длинами ребер.

Суть в том, что отдельно взятый муравей мало что может, поскольку он способен выполнять только максимально простые задачи. Но при большом числе других таких муравьев они могут выступать самостоятельными вычислительными единицами. Муравьи используют непрямой обмен информацией через окружающую среду посредством феромона.

Пусть муравей имеет следующие характеристики:
\begin{enumerate}
	\item зрение --- способность определить длину ребра;
	\item память --- способность запомнить пройденный маршрут;
	\item обоняние --- способность чуять феромон.
\end{enumerate}


Также введем функцию, характеризующую привлекательность ребра, определяемую благодаря зрению:

\begin{equation}
	\label{d_func}
	\eta_{ij} = 1 / D_{ij},
\end{equation}
где $D_{ij}$ — расстояние от текущего пункта $i$ до заданного пункта $j$.


Также понадобится формула вычисления вероятности перехода в заданную точку:
\begin{equation}
	\label{posib}
	p_{k,ij} = \begin{cases}
		\frac{\eta_{ij}^{\alpha}\cdot\tau_{ij}^{\beta}}{\sum_{q\notin J_k} \eta^\alpha_{iq}\cdot\tau^\beta_{iq}}, j \notin J_k, \\
		0, j \in J_k,
	\end{cases}
\end{equation}
где $a$ --- параметр влияния длины пути, $b$ --- параметр влияния феромона, $\tau_{ij}$ --- количество феромонов на ребре $ij$, $\eta_{ij}$ --- привлекательность ребра $ij$, $J_k$ --- список посещенных за текущий день портов.

После завершения движения всех муравьев (ночью, перед наступлением следующего дня), феромон обновляется по следующей формуле:
\begin{equation}
	\label{update_phero_1}
	\tau_{ij}(t+1) = \tau_{ij}(t)\cdot(1-p) + \Delta \tau_{ij}(t).
\end{equation}

При этом величина, на которую меняется количество феромона, считается по формуле:
\begin{equation}
	\label{update_phero_2}
	\Delta \tau_{ij}(t) = \sum_{k=1}^N \Delta \tau^k_{ij}(t),
\end{equation}
где
\begin{equation}
	\label{update_phero_3}
	\Delta\tau^k_{ij}(t) = \begin{cases}
		Q/L_{k}, \textrm{ребро посещено муравьем $k$ в текущий день $t$,} \\
		0, \textrm{иначе}.
	\end{cases}
\end{equation}
$Q$ настраивает концентрацию феромонов и должно быть соразмерно длине лучшего найденного пути.

Поскольку вероятность (\ref{posib}) перехода в заданную точку не должна быть равна нулю, необходимо обеспечить неравенство $\tau_{ij} (t)$ нулю посредством введения дополнительного минимально возможного значения феромона $\tau_{min}$ и в случае, если $\tau_{ij} (t+1)$ принимает значение, меньшее $\tau_{min}$, откатывать феромон до этой величины.


Для выбора пути используется набор ограничений.
\begin{enumerate}
	\item Каждый муравей имеет список запретов --- список уже посещенных портов (вершин графа).
	\item Муравьиное зрение отвечает за эвристическое желание посетить вершину.
	\item Муравьиное обоняние отвечает за ощущение феромона на определенном пути (ребре). При этом количество феромона на пути (ребре) в день $t$ обозначается как $\tau_{i, j} (t)$.
	\item После прохождения определенного ребра муравей откладывает на нем некоторое количество феромона, которое показывает оптимальность сделанного выбора, количество вычисляется по формуле~\eqref{update_phero_3}.
\end{enumerate}

Следует упомянуть, что сумма коэффициентов $\alpha$ и $\beta$ равна 1.

\section*{Элитные муравьи}
В данной реализации муравьиного алгоритма используются элитные муравьи --- после дня жизни они откладывают феромоны на лучшем пути, словно прошли обычные муравьи.

\section*{Вывод}

В данном разделе была рассмотрена задача коммивояжера, а также способы её решения --- полный перебор и методе на основе муравьиного алгоритма.
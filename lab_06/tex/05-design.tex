\chapter{Конструкторская часть}

В данном разделе будут представлены схемы алгоритма полного перебора и муравьиного алгоритма.

\section{Требования к ПО}

К программе предъявлены ряд требований:

\begin{itemize}[label=---]
	\item программа должна получать на вход матрицу смежности, для которой можно будет выбрать один из алгоритмов поиска оптимальных путей (полным перебором или муравьиным алгоритмом);
	\item программа должна позволять пользователю определять коэффициенты и количество дней для метода на основе муравьиного алгоритма;
	\item программа должна давать возможность получить минимальную сумму пути, а также сам путь, используя один из алгоритмов.
\end{itemize}

\section{Разработка алгоритмов}

На рисунке~\ref{img:full-comb} представлена схема алгоритма полного перебора путей.
На рисунках~\ref{img:ants-1} и \ref{img:ants-2}  представлена схема муравьиного алгоритма.

\includeimage
{full-comb} % Имя файла без расширения (файл должен быть расположен в директории inc/img/)
{f} % Обтекание (без обтекания)
{h} % Положение рисунка (см. figure из пакета float)
{0.8\textwidth} % Ширина рисунка
{Схема алгоритма полного перебора путей} % Подпись рисунк

\includeimage
	{ants-1} % Имя файла без расширения (файл должен быть расположен в директории inc/img/)
	{f} % Обтекание (без обтекания)
	{h} % Положение рисунка (см. figure из пакета float)
	{1\textwidth} % Ширина рисунка
	{Схема муравьиного алгоритма (часть 1)} % Подпись рисунк


\includeimage
	{ants-2} % Имя файла без расширения (файл должен быть расположен в директории inc/img/)
	{f} % Обтекание (без обтекания)
	{h} % Положение рисунка (см. figure из пакета float)
	{0.8\textwidth} % Ширина рисунка
	{Схема муравьиного алгоритма (часть 2)} % Подпись рисунк


\includeimage
{find-pos} % Имя файла без расширения (файл должен быть расположен в директории inc/img/)
{f} % Обтекание (без обтекания)
{h} % Положение рисунка (см. figure из пакета float)
{1\textwidth} % Ширина рисунка
{Схема алгоритма нахождения массива вероятностей переходов в непосещенные порты} % Подпись рисунк

\includeimage
{update} % Имя файла без расширения (файл должен быть расположен в директории inc/img/)
{f} % Обтекание (без обтекания)
{h} % Положение рисунка (см. figure из пакета float)
{0.7\textwidth} % Ширина рисунка
{Схема алгоритма обновления матрицы феромонов} % Подпись рисунк


\includeimage
{rand-choice} % Имя файла без расширения (файл должен быть расположен в директории inc/img/)
{f} % Обтекание (без обтекания)
{h} % Положение рисунка (см. figure из пакета float)
{0.8\textwidth} % Ширина рисунка
{Схема алгоритма выбора следующего порта} % Подпись рисунк



\section{Описание используемых типов данных}
При реализации алгоритмов будут использованы следующие типы данных:
\begin{itemize}[label=---]
	\item размер матрицы смежности --- целое число;
	\item имя файла --- строка;
	\item матрица смежности --- матрица целых чисел.
\end{itemize}

\section*{Вывод}
В данном разделе была рассмотрена задача коммивояжера, а также метод полного перебора для её решения и метода на основе муравьиного алгоритма. Были представлены требования к разрабатываемому программному обеспечению.
\chapter*{Введение}
\addcontentsline{toc}{chapter}{Введение}

Задача коммивояжёра --- задача транспортной логистики, отрасли, занимающейся планированием транспортных перевозок.
Коммивояжёру, чтобы распродать нужные и не очень нужные в хозяйстве товары, следует объехать $n$ пунктов и в конце концов вернуться в исходный пункт.
Требуется определить наиболее выгодный маршрут объезда. В качестве меры выгодности маршрута может служить суммарное время в пути, суммарная стоимость дороги, или, в простейшем случае, длина маршрута~\cite{comi}.

Муравьиный алгоритм --- полиномиальный алгоритм для нахождения приближённых решений задачи коммивояжёра, а также решения аналогичных задач поиска маршрутов на графах~\cite{comi}.

Суть подхода заключается в анализе и использовании модели поведения муравьёв, ищущих пути от колонии к источнику питания, и представляет собой метаэвристическую оптимизацию.

Целью данной лабораторной работы является параметризация метода решения задачи коммивояжера на основе муравьиного метода.

Для поставленной цели необходимо выполнить следующие задачи.
\begin{enumerate}
	\item Описать задачу коммивояжера.
	\item Описать методы решения задачи коммивояжера --- метод полного перебора и метод на основе муравьиного алгоритма.
	\item Привести схемы муравьиного алгоритма и алгоритма, позволяющего решить задачу коммивояжера методом полного перебора.
	\item Разработать и реализовать программный продукт, позволяющий решить задачу коммивояжера исследуемыми методами.
	\item Сравнить по времени метод полного перебора и метод на основе муравьиного алгоритма.
	\item Описать и обосновать полученные результаты в отчете о выполненной лабораторной работе.
\end{enumerate}

Выданный индивидуальный вариант для выполнения лабораторной работы:
\begin{itemize}
	\item ориентированый граф;
	\item с элитными муравьев;
	\item маршрут без возврата в исходный порт.
\end{itemize}
\chapter{Исследовательская часть}

\section{Технические характеристики}

Технические характеристики устройства, на котором выполнялись замеры по времени, представлены далее.

\begin{itemize}
	\item Процессор: Ryzen 5 4600H, тактовая частота ЦПУ 3.0 ГГц, максимальная частота процессора 4.0 ГГц~\cite{ryzen}.
	\item Оперативная память: 16 ГБайт.
	\item Операционная система: Windows 10 Pro 64-разрядная система \cite{windows}.
\end{itemize}

При замерах времени ноутбук был включен в сеть электропитания и был нагружен только системными приложениями.

\section{Демонстрация работы программы}


На рисунке~\ref{img:demonstration} представлено визуальное представление работы разработанного программного обеспечения. 
В частности, продемонстрированы результаты вычислений, выполненных алгоритмами поиска расстояний Левенштейна и Дамерау~---~Левенштейна для двух строк: <<акбачок>> и <<кабачок>>. 
В рамках этой демонстрации представлены матрицы, отражающие промежуточные результаты для соответствующих алгоритмов.

\newpage
\includeimage
	{demonstration} % Имя файла без расширения (файл должен быть расположен в директории inc/img/)
	{f} % Обтекание (без обтекания)
	{h} % Положение рисунка (см. figure из пакета float)
	{0.7\textwidth} % Ширина рисунка
	{Демонстрация работы программы при поиске расстояний Левенштейна и Дамерау~---~Левенштейна} % Подпись рисунка


\section{Временные характеристики}

В таблице~\ref{tbl:time} представлен результат замеров времени реализованных алгоритмов.
Замеры проводились для одинаковых длин строк. 
Строки были сгенерированы путем последовательного добавления символа, в конец строки. 

Используя значения из таблицы~\ref{tbl:time}, был построен график~\ref{img:graph1} для лучшей визуализации эффективности алгоритмов.

\newpage
\begin{table}[h]
	\small
	\begin{center}
		\begin{threeparttable}
			\caption{Результат замеров времени реализованных алгоритмов в тактах процессора}
			\label{tbl:time}
			\begin{tabular}{|c|c|c|c|c|}
				\hline
				& \multicolumn{4}{c|}{\bfseries Время, нс} \\ \cline{2-5}
				& \multicolumn{1}{c|}{\bfseries Левенштейн}
				& \multicolumn{3}{c|}{\bfseries Дамерау~---~Левенштейн} \\ \cline{2-5}
				\bfseries Длина (символ) & \bfseries Итеративный & \bfseries Итеративный & \multicolumn{2}{c|}{\bfseries Рекурсивный} \\ \cline{4-5}
				& & & \bfseries Без кеша & \bfseries С кешем
				\csvreader{inc/csv/round_time.csv}{}
				{\\\hline \csvcoli & \csvcolii & \csvcoliii & \csvcoliv & \csvcolv} \\
				\hline
			\end{tabular}			
		\end{threeparttable}
	\end{center}
\end{table}

% todo
\includeimage
	{graph1} % Имя файла без расширения (файл должен быть расположен в директории inc/img/)
	{w} % Обтекание (без обтекания)
	{h} % Положение рисунка (см. figure из пакета float)
	{1\textwidth} % Ширина рисунка
	{Результат замеров времени реализованных алгоритмов} % Подпись рисунка
\clearpage


\section{Характеристики по памяти}

Введем следующие обозначения:

\begin{itemize}
	\item $m$~--- длина строки $S_1$;
	\item $n$~--- длина строки $S_2$;
	\item $\text{size}(v)$~--- функция, вычисляющая размер входного параметра $v$ в байтах;
	\item $char$~--- тип данных, используемый для хранения символа строки;
	\item $int$~--- целочисленный тип данных.
\end{itemize}

Теоретическая оценка объема используемой памяти для итеративной реализации алгоритма поиска расстояния Левенштейна:

\begin{multline}
	M_{LevIter} = m \cdot n \cdot \text{size}(int) +  2 \cdot  \text{size}(char * ) + 3 \cdot \text{size}(int) + 2 \cdot \text{size}(int)
\end{multline}

где $(m \cdot n) \cdot \text{size}(int)$~--- размер матрицы,
\newline $2 \cdot \text{size}(char *)$~--- размер двух указателей входных строк,
\newline $2 \cdot \text{size}(int)$~--- размер переменных, хранящих длину строк,
\newline $3 \cdot \text{size}(int)$~--- размер дополнительных переменных.

Для итеративного алгоритма поиска расстояния Дамерау~---~Левенштейна теоретическая оценка объема используемой памяти идентична.

Теоретическая оценка объема затраченной памяти для рекурсивных реализаций алгоритма нахождения расстояния Дамерау~---~Левенштейна.

Расчет объема памяти, используемой каждым вызовом функции поиска расстояния Дамерау~---~Левенштейна:

\begin{equation}
	M_{call} = 2 \cdot \text{size}(char *) + 2 \cdot \text{size}(int) + \text{size}(int) + 8
\end{equation}
где $2 \cdot \text{size}(char *)$~--- двух указателей входных строк,
\newline $2 \cdot \text{size}(int)$~--- размер двух входных строк,
\newline $\text{size}(int)$~--- размер вспомогательной переменной,
\newline 8 байт~--- адрес возврата.

Максимальная глубина стека вызовов при рекурсивной реализации равна сумме длин входящих строк, поэтому максимальный расход памяти равен

\begin{equation}
	M_{DLRec} = (m + n) \cdot M_{call}
\end{equation}
где $m + n$~--- максимальная глубина стека вызовов,
\newline $M_{call}$~--- затраты по памяти для одного рекурсивного вызова.

Рекурсивная реализация алгоритма поиска расстояния Дамерау~---~Левенштейна с кэшированием для хранения промежуточных значений использует матрицу кэш, размер которой можно рассчитать следующим образом:

\begin{equation}
	M_{cache} = (n \cdot m) \cdot \text{size}(int) 
\end{equation}
где $(n \cdot m)$~--- количество элементов в кэше,

Следовательно, теоретическая оценка объема используемой памяти для рекурсивного алгоритма нахождения расстояния Дамерау~---~Левенштейна с использованием кэша:

\begin{equation}
	M_{DLRecCache} = M_{DLRec} + M_{cache}
\end{equation}


В таблице~\ref{tbl:meme}, приведен расчет затраты памяти в байтах для различных алгоритмов в зависимости от длины строки.
Пусть тип $char$ равен 1 байту, тип $int$ равен 4 байтам, а указатель на строку  $char *$ равен 8 байтам.

\begin{table}[h]
	\small
	\begin{center}
		\begin{threeparttable}
			\caption{Затраты памяти для различных алгоритмов}
			\label{tbl:meme}
			\begin{tabular}{|c|c|c|c|c|}
				\hline
				& \multicolumn{4}{c|}{\bfseries Память, байт} \\ \cline{2-5}
				& \multicolumn{1}{c|}{\bfseries Левенштейн}
				& \multicolumn{3}{c|}{\bfseries Дамерау~---~Левенштейн} \\ \cline{2-5}
				\bfseries Длина (символ) & \bfseries Итеративный & \bfseries Итеративный & \multicolumn{2}{c|}{\bfseries Рекурсивный} \\ \cline{4-5}
				& & & \bfseries Без кеша & \bfseries С кешем
				\csvreader{inc/csv/meme.csv}{}
				{\\\hline \csvcoli & \csvcolii & \csvcoliii & \csvcoliv & \csvcolv} \\
				\hline
			\end{tabular}			
		\end{threeparttable}
	\end{center}
\end{table}


\section{Вывод}

В результате исследования реализуемых алгоритмов по времени выполнения можно сделать следующие выводы:

Итеративные реализации алгоритмов нахождения расстояния Левенштейна и Дамерау~---~Левенштейна по итогам исследования показали наименьшее время (см.~рис.~\ref{img:graph1}), но из-за дополнительной операцией в алгоритме Дамерау~---~Левенштейна, его реализация медленнее в $1.25$ раза при длине строк 11 символов.

Реализация рекурсивного алгоритма нахождения расстояния Дамерау~---~Левенштейна значительно проигрывает остальным алгоритмам (см.~рис.~\ref{img:graph1}).
Такая значительная разница в производительности обусловлена проблемой повторных вычислений, с которой сталкиваются рекурсивные алгоритмы \cite{recs}. 
При обработке слов длиной в 11 символов эта проблема становится заметной, так как количество повторных вычислений значительно увеличивается.
Внедрение кэширование в алгоритм нахождения расстояния Дамерау~---~Левенштейна привело к решению проблемы повторных вычислений в рекурсивных алгоритмах.

Полученные результаты могут зависеть от характеристик используемой вычислительной системы. Возможны вариации результатов при выполнении замеров на различных компьютерах.

В результате исследования алгоритмов по затрачиваемой памяти
можно сделать следующие выводы:

Итеративные алгоритмы и рекурсивный алгоритм с кэшированием требуют больше памяти по сравнению с рекурсивным. 
В реализациях, использующих матрицу, максимальный размер используемой памяти увеличивается пропорционально произведению длин строк, в то время как у рекурсивного алгоритма без кэширования объем затрачиваемой памяти увеличивается пропорционально сумме длин строк.
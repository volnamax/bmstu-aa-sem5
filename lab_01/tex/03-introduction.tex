\chapter*{ВВЕДЕНИЕ}
\addcontentsline{toc}{chapter}{ВВЕДЕНИЕ}

Целью данной лабораторной работы является изучение, реализация и сравнительный анализ алгоритмов поиска редакционных расстояний Левенштейна и Дамерау~---~Левенштейна.

\textbf{Расстояние Левенштейна} (редакционное расстояние) представляет собой метрику, которая измеряет разницу между двумя строками. Это определяет минимальное количество односимвольных операций (вставка, удаление и замена символа), необходимых для преобразования одной строки в другую~\cite{analysis-lev-damlev}.

\textbf{Расстояние Дамерау~---~Левенштейна} представляет собой расширенную версию расстояния Левенштейна, включающую дополнительную операцию --- транспозицию. Данная операция позволяет учесть случаи перестановки двух соседних символов в строке~\cite{analysis-lev-damlev}.

Расстояние Левенштейна и его модификация расстояние Дамерау~---~Левенштейна имеют широкий спектр применений, например, в лингвистике (сравнение текстовых файлов или исправление ошибок в словах), в биоинформатике (сравнения генов, белков и хромосом)~\cite{prog-impl-lev, analysis-lev-damlev}.


Задачи лабораторной работы:
\begin{enumerate}
	\item описать расстояния Левенштейна и Дамерау~---~Левенштейна;
	\item описать алгоритмы поиска расстояний Левенштейна и Дамерау~---~Левенштейна;
	\item разработать программное обеспечение, включающее в себя нерекурсивный алгоритм поиска расстояния Левенштейна и нерекурсивный алгоритм поиска расстояния Дамерау~---~Левенштейна, а также рекурсивный алгоритм поиска расстояния Дамерау~---~Левенштейна и рекурсивный с кешированием алгоритм поиска расстояния Дамерау~---~Левенштейна;
	\item провести сравнительный анализ времени выполнения реализаций алгоритмов и занимаемой памяти.
\end{enumerate}

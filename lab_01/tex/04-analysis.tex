\chapter{Аналитический раздел}


\section{Расстояние Левенштейна}

Расстояние Левенштейна между двумя строками --- это минимальное количество редакторских операций: вставка (I от англ. insert), замена (R от англ. replace) и удаление (D от англ. delete), необходимых для преобразования одной строки в другую.
Стоимость каждой операции может различаться в зависимости от вида операции~\cite{levenshtein}:

\begin{enumerate}
	\item $w(a, b) = 1$, $a \neq b$ --- цена замены символа $a$ на $b$;
	\item $w(\lambda, b) = 1$ --- цена вставки символа $b$;
	\item $w(a, \lambda) = 1$ --- цена удаления символа $a$.
\end{enumerate}

Введем понятие совпадения символов --- M (от англ. match). 
Его стоимость будет равна 0, то есть $w(a, a) = 0$~\cite{analysis-lev-damlev}.

Пусть  имеется две строки $S_{1}$ и $S_{2}$ длиной $M$ и $N$ соответственно, тогда редакционное расстояние $d(S_{1}, S_{2})$ можно подсчитать по следующей рекуррентной формуле \cite{prog-impl-lev}:

\begin{equation}
	\label{eq:L}
	 d(S_{1}, S_{2}) = D(M, N) = \begin{cases}
		
		0, &\text{i = 0, j = 0}\\
		i, &\text{j = 0, i > 0}\\
		j, &\text{i = 0, j > 0}\\
		\min ( 
		 D(i, j-1) + 1,\\
		\qquad D(i-1, j) + 1, &\text{i > 0, j > 0}\\
		\qquad D(i-1, j-1) + \\ 
		\qquad m(S_{1}[i], S_{2}[j]) ).
	\end{cases}
\end{equation}

где сравнение символов строк $S_1$ и $S_2$ рассчитывается как

\begin{equation}
	\label{eq:m}
	m(a, b) = \begin{cases}
		0 &\text{если a = b,}\\
		1 &\text{иначе.}
	\end{cases}
\end{equation}

\subsection{Нерекурсивный алгоритм нахождения расстояния Левенштейна}

Под нерекурсивным алгоритмом нахождения расстояния Левинштейна понимается итеративная реализация алгоритма.

В качестве структуры данных для хранения промежуточных значений можно использовать матрицу, имеющую размеры $(N + 1) \times (M + 1)$.

Пусть $M$ --- длина строки $S_{1}$, $N$ --- длина строки $S_{2}$. 
$S_{1}[1...i]$ --- подстрока $S_{1}$ с длиной $i$ символов, начиная с первого, $S_{2}[1...j]$ --- подстрока $S_{2}$ длиной $j$ символов, начиная с первого. 
Каждая ячейка матрицы, обозначенная как $mtr[i, j]$ содержит значение $D(S1[1...i], S2[1...j])$. Вся матрица заполняется в соответствии с формулой~(\ref{eq:L}). 

\section{Расстояние Дамерау~---~Левенштейна}
Расстояние Дамерау~---~Левенштейна между двумя строками, представляет собой минимальное количество операций, таких как вставка, удаление, замена символа и перестановка соседних символов (транспозиция), чтобы преобразовать одну строку в другую.
Это является модифицированной версией расстояния Левенштейна, в которую была добавлена операция перестановки, также известная как транспозиция~\cite{analysis-lev-damlev}.

Расстояние Дамерау~---~Левенштейна может быть вычислено по рекуррентной формуле \cite{levenshtein}:

\begin{equation}
	\label{eq:DL}
	D(M, N) = 
	\begin{cases}
		0, &\text{i = 0, j = 0,}\\
		i, &\text{j = 0, i > 0,}\\
		j, &\text{i = 0, j > 0,}\\
		 \min (  D(i, j - 1) + 1,\\
			\qquad D(i - 1, j) + 1,\\
			\qquad D(i - 1, j - 1) + m(S_{1}[i], S_{2}[j]), \\
			\qquad D(i - 2, j - 2) + 1 ),
		& \begin{aligned}
			& \text{если i > 1, j > 1}, \\
			& S_{1}[i] = S_{2}[j - 1], \\
			& S_{1}[i - 1] = S_{2}[j], \\
		\end{aligned}\\
		
			\min ( D(i, j - 1) + 1,\\
			\qquad  D(i - 1, j) + 1, \\
			\qquad  D(i - 1, j - 1) + m(S_{1}[i], S_{2}[j]) ) & \text{, иначе.}
	\end{cases}
\end{equation}

\subsection{Рекурсивный алгоритм нахождения расстояния Дамерау~---~Левенштейна}

Рекурсивный алгоритм реализует формулу (\ref{eq:DL}), функция $D$ составлена таким образом, что верно следующее.
\begin{enumerate}
	\item Перевод из пустой строки в пустую не требует действий.
	\item Преобразование пустой строки в строку $S_{1}$ или наоборот требует $|S_{1}|$ операций.
	\item Преобразование строки $S_{1}$ в $S_{2}$ включает комбинацию операций: удаление, вставка и, если применимы, замена и транспозиция. Порядок операций не важен.	
\end{enumerate}

Пологая, что $S_{1}'$  --- это $S_{1}$ без последнего символа и $S_{1}''$  --- это $S_{1}$ без последних двух символов. 
Для строки $S_{2}$ аналогичные обозначения.
Тогда, минимальной ценой преобразования строки $S_{1}$ в строку $S_{2}$, будет минимальное значение из следующих вариантов (при условии, что они применимы).

\begin{enumerate}
	\item Преобразуем $S_{1}'$ в $S_{2}$, а затем удаляем последний символ из $S_{1}'$.
	\item Преобразуем $S_{1}$ в $S_{2}'$, затем добавляем последний символ из $S_{2}$.
	\item Если $S_{1}$ и $S_{2}$ заканчиваются разными символами, меняем последний символ $S_{1}$ на последний символ $S_{2}$, после преобразования $S_{1}'$ в $S_{2}'$.
	\item Если после перестановки двух последних символов $S_{1}$ получаем окончание $S_{2}$, то рассматриваем преобразование $S_{1}''$ в $S_{2}''$.
	\item Если $S_{1}$ и $S_{2}$ заканчиваются одинаковым символом, то рассматриваем преобразование $S_{1}'$ в $S_{2}'$.
\end{enumerate}



\subsection{Рекурсивный алгоритм нахождения расстояния Дамерау~---~Левенштейна с кешированием}

Рекурсивный алгоритм с использованием кэша является улучшенной версией рекурсивного алгоритма нахождения расстояния Дамерау~---~Левенштейна.
Он использует так же формулу (\ref{eq:DL}), но с кешированием. 
Добавление механизма кеширования, позволяет избежать повторные вычисление для подстрок~\cite{analysis-lev-damlev}.
Чтобы реализовать механизм кеширования можно использовать матрицу кеша для сохранения промежуточных результатов. 
Алгоритм последовательно заполняет матрицу $mtr_{M, N}$ значениями $D(i, j)$ расстояний. Размер матрицы кэша равен $(N + 1) \times (M + 1)$.

\subsection{Нерекурсивный алгоритм нахождения расстояния Дамерау~---~Левенштейна}

Итеративный алгоритм нахождения расстояния Дамерау~---~Левенштейна решает проблему с повторными вычислениями у рекурсивного алгоритма~\cite{analysis-lev-damlev}. 
	
В качестве структуры данных для хранения промежуточных значений можно использовать матрицу, имеющую  размеры $(N + 1) \times (M + 1)$. 
Каждая ячейка этой матрицы, содержит значение $D(S1[1...i], S2[1...j])$. 
Вся матрица заполняется в соответствии с формулой~(\ref{eq:DL}). 

\section*{Вывод}

В представленном разделе описаны: расстояния Левенштейна и Дамерау~---~Левенштейна, алгоритмы поиска расстояний Левенштейна и Дамерау~---~Левенштейна. Алгоритмы характеризуются рекуррентными формулами~(\ref{eq:L}, \ref{eq:DL}), что позволяет реализовывать два подхода рекурсивный и итеративный~\cite{analysis-lev-damlev}. 
В качестве входных данных алгоритмы обрабатывают строковые последовательности, содержащие символы как кириллицы, так и латиницы, при этом алгоритмическая конструкция предусматривает корректную обработку пустых строковых последовательностей.
\chapter*{ЗАКЛЮЧЕНИЕ}
\addcontentsline{toc}{chapter}{ЗАКЛЮЧЕНИЕ}

Цели работы достигнуты: изучены, реализованы и проведен сравнительный анализ алгоритмов поиска редакционных расстояний Левенштейна и Дамерау~---~Левенштейна.

В ходе выполнения лабораторной работы были решены все задачи: 
\begin{enumerate}
	\item описаны расстояния Левенштейна и Дамерау~---~Левенштейна;
	\item описаны алгоритмы поиска расстояний Левенштейна и Дамерау~---~Левенштейна;
	\item разработано программное обеспечение, включающее в себя нерекурсивный алгоритм поиска расстояния Левенштейна и нерекурсивный алгоритм поиска расстояния Дамерау~---~Левенштейна, а также рекурсивный алгоритм поиска расстояния Дамерау~---~Левенштейна и рекурсивный с кешированием алгоритм поиска расстояния Дамерау~---~Левенштейна;
	\item проведен сравнительный анализ времени выполнения реализаций алгоритмов и занимаемой памяти.
	\begin{itemize}
		\item При обработке строк длиной до 11 символов разница между временем выполнения нерекурсивных реализаций алгоритмов Левенштейна и Дамерау~---~Левенштейна незначительна, но из-за дополнительной операции в алгоритме Дамерау~---~Левенштейна, его реализация работает медленнее, а именно в $1.25$ раза при длине слов 11 символов.
		\item Рекурсивный алгоритм поиска расстояния Дамерау~---~Левенштейна выполняется на порядок дольше, чем тот же алгоритм, использующий кэширование.
		\item Время работы матричного и рекурсивного с кэшированием алгоритмов поиска расстояния Дамерау~---~Левенштейна приблизительно равно.
		\item Итеративные алгоритмы и рекурсивный алгоритм с кэшированием требуют больше памяти по сравнению с рекурсивным. 
		В реализациях, использующих матрицу, максимальный размер используемой памяти увеличивается пропорционально произведению длин строк, в то время как у рекурсивного алгоритма без кэширования объем затрачиваемой памяти увеличивается пропорционально сумме длин строк.
	\end{itemize}
\end{enumerate}
\chapter{Технологический раздел}

В данном разделе будут приведены средства реализации, листинги кода реализации алгоритмов и функциональные тесты.

\section{Средства реализации}

В качестве языка программирования для реализации лабораторной работы был выбран $Python$~\cite{python-lang}. 
Данный выбор обусловлен наличием необходимых библиотек для проведения точного замера времени. 
Для измерения процессорного времени выполнения была выбрана функция $process\_time$ из модуля $time$~\cite{python-time}.

\section{Сведения модулей программы}

Программа разбита на следующие модули.

\begin{itemize}
	\item \texttt{main.py} --- файл, который включает в себя точку входа программы, из которой осуществляется вызов алгоритмов через разработанный интерфейс.
	\item \texttt{algorithms.py} --- файл содержит функции поиска расстояния Левенштейна и Дамерау~---~Левенштейна.
	\item \texttt{measurement.py} --- в файле присутствуют функции для измерения процессорного времени выполнения реализаций алгоритмов поиска расстояния~Левенштейна~и~Дамерау-Левенштейна. 
\end{itemize}

\section{Реализация алгоритмов}

В листингаx~\ref{lst:lev_mtr.py}~--~\ref{lst:dameray_lev_rec_cache.py} приведены реализации алгоритмов поиска расстояний Левенштейна (только нерекурсивный алгоритм) и Дамерау~---~Левенштейна (нерекурсивный, рекурсивный и рекурсивный с кешированием).

\newpage
\includelistingpretty
	{lev_mtr.py} % Имя файла с расширением (файл должен быть расположен в директории inc/lst/)
	{python} % Язык программирования (необязательный аргумент)
	{Функция нахождения расстояния Левенштейна с использованием матрицы} % Подпись листинга

	

\newpage
\includelistingpretty
	{dameray_lev_mtr.py} % Имя файла с расширением (файл должен быть расположен в директории inc/lst/)
	{python} % Язык программирования (необязательный аргумент)
	{Функция нахождения расстояния Дамерау~---~Левенштейна с использованием матрицы} % Подпись листинга
	

\newpage
\includelistingpretty
	{dameray_lev_recurce.py} % Имя файла с расширением (файл должен быть расположен в директории inc/lst/)
	{python} % Язык программирования (необязательный аргумент)
	{Рекурсивная функция нахождения расстояния Дамерау~---~Левенштейна} % Подпись листинга


\newpage
\includelistingpretty
	{dameray_lev_rec_cache.py} % Имя файла с расширением (файл должен быть расположен в директории inc/lst/)
	{python} % Язык программирования (необязательный аргумент)
	{Рекурсивная функция нахождения расстояния Дамерау~---~Левенштейна с "кэшированием"} % Подпись листинга

\section{Функциональные тесты}

В таблице~\ref{tbl:func_tests} приведены функциональные тесты для алгоритмов вычисления расстояний Левенштейна и Дамерау~---~Левенштейна. 
Все тесты пройдены успешно.

\begin{table}[ht]
	\small
	\begin{center}
		\begin{threeparttable}
			\caption{Функциональные тесты}
			\label{tbl:func_tests}
			\begin{tabular}{|c|c|c|c|c|c|}
				\hline
				\multicolumn{2}{|c|}{\bfseries Входные данные}
				& \multicolumn{4}{c|}{\bfseries Расстояние и алгоритм} \\ 
				\hline 
				&
				& \multicolumn{1}{c|}{\bfseries Левенштейна} 
				& \multicolumn{3}{c|}{\bfseries Дамерау~---~Левенштейн} \\ \cline{3-6}
				
				\bfseries Строка 1 & \bfseries Строка 2 & \bfseries Итеративный & \bfseries Итеративный
				
				& \multicolumn{2}{c|}{\bfseries Рекурсивный} \\ \cline{5-6}
				& & & & \bfseries Без кеша & \bfseries С кешем \\
				\hline
				окно & окна & 1 & 1 & 1 & 1 \\
				\hline
				mama & mama & 0 & 0 & 0 & 0 \\
				\hline
				мама & папа & 2 & 2 & 2 & 2 \\
				\hline
				кот & скат & 2 & 2 & 2 & 2 \\
				\hline
				друзья & рдузия & 3 & 2 & 2 & 2 \\
				\hline
				вагон & гонки & 4 & 4 & 4 & 4 \\
				\hline
				бар & раб & 2 & 2 & 2 & 2 \\
				\hline
				слон & стол & 2 & 2 & 2 & 2 \\
				\hline
				a & b & 1 & 1 & 1 & 1 \\
				\hline
			\end{tabular}	
		\end{threeparttable}
	\end{center}
\end{table}

\section*{Вывод}
	Была создана итеративная реализация алгоритма нахождения расстояния Левенштейна, а также реализованы алгоритмы нахождения расстояния Дамерау~---~Левенштейна в итеративной, рекурсивной и рекурсивной с применением кеширования формах. Проведено тестирование данных реализаций алгоритмов.
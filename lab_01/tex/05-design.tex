\chapter{Конструкторский раздел}

В этом разделе представлены:  схемы алгоритмов  для вычисления расстояний Левенштейна и Дамерау~---~Левенштейна, описание используемых типов данных и необходимые требования для программного обеспечения.

\section{Требования к программному обеспечению}\label{section:requirements}

К программе предъявляются следующие требования:
\begin{itemize}
	\item на вход подается две строки;
	\item на выходе --- искомое расстояние;
	\item интерфейс для выбора операций;
	\item обработка входящих строк;
	\item поддержка строк, содержащих символы как в латинском, так и в кириллическом алфавитах;
	\item возможность произвести замеры процессорного времени работы реализованных алгоритмов поиска расстояний Левенштейна и Дамерау~---~Левенштейна.
\end{itemize}

\section{Разработка алгоритмов}

На вход алгоритмов подаются строки $X$ и $Y$.

На рисунке~\ref{img:lev} представлена схема алгоритма поиска расстояния Левенштейна.
На рисунках~\ref{img:dam_lev}~--~\ref{img:dam_lev_rec_cache} представлены схемы алгоритмов поиска расстояния Дамерау~---~Левенштейна.

\includeimage
	{lev} % Имя файла без расширения (файл должен быть расположен в директории inc/img/)
	{f} % Обтекание (без обтекания)
	{h} % Положение рисунка (см. figure из пакета float)
	{0.7\textwidth} % Ширина рисунка
	{Схема нерекурсивного алгоритма нахождения расстояния Левенштейна} % Подпись рисунк
	
\clearpage

\includeimage
	{dam_lev} % Имя файла без расширения (файл должен быть расположен в директории inc/img/)
	{f} % Обтекание (без обтекания)
	{h} % Положение рисунка (см. figure из пакета float)
	{0.9\textwidth} % Ширина рисунка
	{Схема нерекурсивного алгоритма нахождения расстояния Дамерау~---~Левенштейна} % Подпись рисунк
\clearpage

\includeimage
	{dam_lev_rec} % Имя файла без расширения (файл должен быть расположен в директории inc/img/)
	{f} % Обтекание (без обтекания)
	{h} % Положение рисунка (см. figure из пакета float)
	{1\textwidth} % Ширина рисунка
	{Схема рекурсивного алгоритма нахождения расстояния Дамерау~---~Левенштейна} % Подпись рисунк
\clearpage

\includeimage
	{dam_lev_rec_cache} % Имя файла без расширения (файл должен быть расположен в директории inc/img/)
	{f} % Обтекание (без обтекания)
	{h} % Положение рисунка (см. figure из пакета float)
	{1\textwidth} % Ширина рисунка
	{Схема рекурсивного алгоритма нахождения расстояния Дамерау~---~Левенштейна с кешированием} % Подпись рисунк
\clearpage

\section{Описание используемых типов данных}

При реализации алгоритмов использованы следующие структуры данных:

\begin{itemize}
	\item строка --- символьный массив, размером длины строки;
	\item матрица --- двумерный массив с целыми значениями.
\end{itemize}

\section*{Вывод}

В данном разделе на основе аналитической части были построены схемы
требуемых алгоритмов, выбраны используемые типы данных.
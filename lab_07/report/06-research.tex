\chapter{Исследовательская часть}

В данном разделе приведены результаты подсчета количества сравнений при поиске элементов в худшем и лучшем случаях.

\section{Технические характеристики}

Технические характеристики устройства, на котором выполнялись замеры по времени, представлены далее.

\begin{itemize}
	\item Процессор: Ryzen 5 4600H, 6 процессорных ядер архитектуры Zen 2 и 12 потоков, работающих на базовой частоте в 3.0 ГГц (до 4.0 ГГц в Turbo режиме), 12 логических ядер~\cite{ryzen}
	\item Оперативная память: 16 ГБайт.
	\item Операционная система: Windows 10 Pro 64-разрядная система \cite{windows}.
\end{itemize}

При замерах времени ноутбук был включен в сеть электропитания и был нагружен только системными приложениями.

\section{Демонстрация работы программы}

На рисунке \ref{img:prog_demo} продемонстрирована работа программы для случая, когда пользователь выбрал пункт 1 <<Выполнить поиск элемента алгоритмом бинарного поиска>>, ввел массив $[10, 20, 30, 40, 50]$ и запросил поиск элемента $50$.
Далее пользователь выбрал тот же пункт, ввел массив $[1, 2, 3]$ и запросил поиск элемента $-1$, которого нет в массиве.

\includeimage
	{prog_demo}
	{f}
	{H}
	{1\textwidth}
	{Демонстрация работы программы}

\section{Количество сравнений}

Исследование реализуемых алгоритмов по количеству выполняемых сравнений производилось 2 раза:
\begin{enumerate}
	\item при варьировании числа элементов в массиве $512, 1024, 2048, 4096, 8192$ в лучшем и худших случаях;
	\item при варьировании числа элементов в массиве $513, 1025, 2049, 4097, 8193$ в лучшем и худших случаях;
\end{enumerate}

На рисунке~\ref{img:barplot_best} изображены результаты исследования для лучшего случая.

\includeimage
	{barplot_best}
	{f}
	{H}
	{1\textwidth}
	{Сравнение количества сравнений при работе алгоритмов для лучшего случая}
	
На рисунке~\ref{img:barplot_worst} изображены результаты исследования для худшего случая, когда искомого элемента нет.

\includeimage
	{barplot_worst}
	{f}
	{H}
	{1\textwidth}
	{Сравнение количества сравнений при работе алгоритмов для худшего случая, когда искомого элемента нет}
	
На рисунке~\ref{img:barplot_worst_2} изображены результаты исследования для худшего случая, когда искомый элемент последний.

\includeimage
{barplot_worst_2}
{f}
{H}
{1\textwidth}
{Сравнение количества сравнений при работе алгоритмов для худшего случая, когда искомый элемент последний}

\section{Вывод}
При нечетных размерностях алгоритм поиска с одним сравнением выполняется за одну операцию сравнения. 
Однако на четных размерностях количество выполняемых сравнений больше, чем у алгоритма, выполняющего поиск за два сравнения с медианным элементом.

Для реализации бинарного поиска, использующей два сравнения с медианным элементом, самым худшим случаем работы является поиск последнего элемента массива, а для реализации с одним сравнением~--- при поиске элемента, которого нет в массиве.

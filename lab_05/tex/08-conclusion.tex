\chapter*{ЗАКЛЮЧЕНИЕ}
\addcontentsline{toc}{chapter}{ЗАКЛЮЧЕНИЕ}


Цель лабораторной работы достигнута описаны параллельные конвейерные  вычислений на основе нативных потоков для иерархической кластеризации документов.

В ходе выполнения лабораторной работы были решены все задачи:
\begin{enumerate}
	\item описана организация конвейерной обработки данных;
	\item описаны алгоритмы иерархической кластеризации.
	\item спроектировано программное обеспечение, позволяющее выполнять конвейерную и последовательную обработку данных;
	\item разработано программное обеспечение, позволяющее выполнять конвейерную и последовательную обработку данных;
	\item проведен сравнительный анализ времени работы конвейерной и последовательной версии.
\end{enumerate} 


В результате исследования реализуемых алгоритмов по времени выполнения можно сделать следующий вывод.
Конвейерная реализация алгоритма более эффективна по времени выполнению при увеличении количества заявок, при 24 заявках конвейерная версия обработки на 45.62\% эффективнее по сравнению с последовательной реализацией алгоритма (см. таблицу~\ref{tbl:1}).
Такой результат объясняется тем, что в конвейерной реализации потоки могут выполнять различные этапы работы параллельно, что позволяет сократить время обработки последовательности заявок.

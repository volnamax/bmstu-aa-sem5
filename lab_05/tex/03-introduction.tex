\chapter*{ВВЕДЕНИЕ}
\addcontentsline{toc}{chapter}{ВВЕДЕНИЕ}


Разработчики архитектуры компьютеров издавна прибегали к методам проектирования, известным под общим названием "совмещение операций", при котором аппаратура компьютера в любой момент времени выполняет одновременно более одной базовой операции~\cite{conveyor}.

Этот метод включает в себя, в частности, такое понятие, как конвейеризация.
Конвейеры широко применяются программистами для решения трудоемких задач, которые можно разделить на этапы, а также в большинстве современных быстродействующих процессоров~\cite{conveyor}.

В качестве операций, выполняющихся на конвейере в данной работе, взяты следующие:
\begin{enumerate}
	\item считывание из файлов \textit{TF-IDF};
	\item кластеризация;
	\item запись результата в файл.
\end{enumerate}

Целью данной лабораторной работы является описание параллельных конвейерных вычислений на основе нативных потоков для иерархической кластеризации документов.

В рамках выполнения работы необходимо решить следующие задачи: 
\begin{enumerate}
	\item описать организацию конвейерной обработки данных;
	\item описать алгоритмы иерархической кластеризации, которые будут использоваться в данной лабораторной работе;
	\item спроектировать программное обеспечение, позволяющее выполнять конвейерную и последовательную обработку данных;
	\item разработать программное обеспечение, позволяющее выполнять конвейерную и последовательную обработку данных;
	\item провести сравнительный анализ времени работы конвейерной и последовательной версии.
\end{enumerate} 

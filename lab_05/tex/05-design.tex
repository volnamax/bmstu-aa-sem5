\chapter{Конструкторский раздел}

В данном разделе будут представлены требования к программному обеспечению, описание используемых типов данных и схемы реализуемых алгоритмов.

\section{Требования к программному обеспечению}

К программе предъявлен ряд требований:
\begin{itemize}
	\item должен присутствовать интерфейс для выбора действий;
	\item считывание данных должно производиться из файла;
	\item результат должен записываться в файл;
	\item должен присутствовать замер реального времени для реализаций
	алгоритмов;
	\item результат замера должен выводится в виде таблицы.
\end{itemize}

\section{Описание используемых типов данных}

При реализации алгоритмов будут использованы следующие структуры и типы данных:
\begin{itemize}
	\item массив символов для хранения терма;
	\item вещественное число для хранения \textit{TF-IDF} терма;
	\item мьютекс~--- примитив синхронизации.
\end{itemize}

\section{Разработка алгоритмов}

На рисунке~\ref{img:hierarchicalClustering} приведена схема дивизимной иерархической кластеризации.
На рисунке~\ref{img:kmeans} приведена схема алгоритма k-средних.
На рисунке~\ref{img:treadMain} приведена схема алгоритма запуска конвейера.
На рисунке~\ref{img:generator} приведена схема обслуживающего устройства, которое создает начальные запросы.
На рисунке~\ref{img:serviceRead} приведена схема обслуживающего устройства, которое считывает \textit{TF-IDF} из файла.
На рисунке~\ref{img:serviceCluster} приведена схема обслуживающего устройства, которое кластеризует документы.
На рисунке~\ref{img:serviceWrite} приведена схема обслуживающего устройства, которое записывает результат в файл.



\includeimage
{hierarchicalClustering} % Имя файла без расширения (файл должен быть расположен в директории inc/img/)
{f} % Обтекание (без обтекания)
{h} % Положение рисунка (см. figure из пакета float)
{1\textwidth} % Ширина рисунка
{Схема дивизимной иерархической кластеризации} % Подпись рисунк
\clearpage


\includeimage
{kmeans} % Имя файла без расширения (файл должен быть расположен в директории inc/img/)
{f} % Обтекание (без обтекания)
{h} % Положение рисунка (см. figure из пакета float)
{0.9\textwidth} % Ширина рисунка
{Схема алгоритма k-средних} % Подпись рисунк
\clearpage


\includeimage
{treadMain} % Имя файла без расширения (файл должен быть расположен в директории inc/img/)
{f} % Обтекание (без обтекания)
{h} % Положение рисунка (см. figure из пакета float)
{0.4\textwidth} % Ширина рисунка
{Схема алгоритма запуска конвейера} % Подпись рисунк2
\clearpage

\includeimage
{generator} % Имя файла без расширения (файл должен быть расположен в директории inc/img/)
{f} % Обтекание (без обтекания)
{h} % Положение рисунка (см. figure из пакета float)
{0.4\textwidth} % Ширина рисунка
{Схема алгоритма обслуживающего устройства, которое создает начальные запросы} % Подпись рисунк
\clearpage

\includeimage
{serviceRead} % Имя файла без расширения (файл должен быть расположен в директории inc/img/)
{f} % Обтекание (без обтекания)
{h} % Положение рисунка (см. figure из пакета float)
{1\textwidth} % Ширина рисунка
{Схема алгоритма обслуживающего устройства, которое считывает \textit{TF-IDF} из файла} % Подпись рисунк
\clearpage

\includeimage
{serviceCluster} % Имя файла без расширения (файл должен быть расположен в директории inc/img/)
{f} % Обтекание (без обтекания)
{h} % Положение рисунка (см. figure из пакета float)
{1\textwidth} % Ширина рисунка
{Схема алгоритма обслуживающего устройства, которое кластеризует документы} % Подпись рисунк
\clearpage


\includeimage
{serviceWrite} % Имя файла без расширения (файл должен быть расположен в директории inc/img/)
{f} % Обтекание (без обтекания)
{h} % Положение рисунка (см. figure из пакета float)
{1\textwidth} % Ширина рисунка
{Схема алгоритма обслуживающего устройства, которое записывает результат в файл} % Подпись рисунк

\section *{Вывод}

В данном разделе были представлены требования к программному обеспечению, описание используемых типов данных и схемы реализуемых алгоритмов.

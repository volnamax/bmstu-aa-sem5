\chapter{Исследовательская часть}

В данном разделе будут приведены демонстрация работы программы, технические характеристики устройства, сравнительный анализ времени выполнения реализуемых алгоритмов.


\section{Демонстрация работы программы}

На рисунке~\ref{img:demka} представлена демонстрация работы разработанного программного обеспечения, а именно показан результат собранной статистики для каждой заявки. На рисунке~\ref{img:demka}, N --- означает номер заявки, Time~1, Time~2, Time~3~---~ означают время выполнения в миллисекундах для обслуживающих устройств, которое считывает \textit{TF-IDF} из файла, которое кластеризует документы и которое записывает результат в файл, соответственно.
Пример входных файлов представлен на рисунке~\ref{img:in}. 
Пример результата кластеризации в виде json файла, представлен на рисунке~\ref{img:out}.

\includeimage
{demka} % Имя файла без расширения (файл должен быть расположен в директории inc/img/)
{f} % Обтекание (без обтекания)
{h} % Положение рисунка (см. figure из пакета float)
{1\textwidth} % Ширина рисунка
{Демонстрация работы программы} % Подпись рисунк

\includeimage
{in} % Имя файла без расширения (файл должен быть расположен в директории inc/img/)
{f} % Обтекание (без обтекания)
{h} % Положение рисунка (см. figure из пакета float)
{0.4\textwidth} % Ширина рисунка
{Пример входного файла} % Подпись рисунк

\includeimage
{out} % Имя файла без расширения (файл должен быть расположен в директории inc/img/)
{f} % Обтекание (без обтекания)
{h} % Положение рисунка (см. figure из пакета float)
{0.2\textwidth} % Ширина рисунка
{Пример результата кластеризации} % Подпись рисунк

\clearpage

\section{Технические характеристики}

Технические характеристики устройства, на котором выполнялись замеры по времени, представлены далее.

\begin{itemize}
	\item Процессор: Ryzen 5 4600H, 6 процессорных ядер архитектуры Zen 2 и 12 потоков, работающих на базовой частоте в 3.0 ГГц (до 4.0 ГГц в Turbo режиме), 12 логических ядер~\cite{ryzen}
	\item Оперативная память: 16 ГБайт.
	\item Операционная система: Windows 10 Pro 64-разрядная система \cite{windows}.
\end{itemize}

При замерах времени ноутбук был включен в сеть электропитания и был нагружен только системными приложениями.


\section{Время выполнения реализаций алгоритмов}

Результаты замеров времени выполнения реализации алгоритма иерархической кластеризации документов в зависимости от числа заявок приведены в таблице~\ref{tbl:1}.
Каждый замер проводился 100 раз, после чего рассчитывалось их среднее арифметическое значение.

На рисунке~\ref{img:graph} изображен график зависимостей времени выполнения реализаций от числа заявок.
\clearpage

\begin{center}
	\begin{longtable}[c]{|c|c|c|}
		\captionsetup{justification=raggedright,singlelinecheck=off}
		\caption{Зависимость времени выполнения (в мс) от количества заявок 
			\label{tbl:1}}
		\\ 
		\hline
		Кол-во заявок & Послед. реализация (мс) & Конвейер. реализация (мс) 
		\csvreader{inc/csv/performance_comparison_data.csv}{}
		{\\ \hline \csvcoli & \csvcolii & \csvcoliii} 
		\\ \hline
	\end{longtable}
\end{center}

\clearpage
\includeimage
{graph} % Имя файла без расширения (файл должен быть расположен в директории inc/img/)
{f} % Обтекание (без обтекания)
{h} % Положение рисунка (см. figure из пакета float)
{1\textwidth} % Ширина рисунка
{График зависимости времени выполнения реализации от числа заявок} % Подпись рисунк


\section*{Вывод}

В результате исследования реализуемых алгоритмов по времени выполнения можно сделать следующий вывод.
Конвейерная реализация алгоритма более эффективна по времени выполнению при увеличении количества заявок, при 24 заявках конвейерная версия обработки на 45.62\% эффективнее по сравнению с последовательной реализацией алгоритма (см. таблицу~\ref{tbl:1}).
Такой результат объясняется тем, что в конвейерной реализации потоки могут выполнять различные этапы работы параллельно, что позволяет сократить время обработки последовательности заявок.



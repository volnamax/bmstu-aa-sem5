\chapter*{ЗАКЛЮЧЕНИЕ}
\addcontentsline{toc}{chapter}{ЗАКЛЮЧЕНИЕ}


Цель работы достигнута, проведен сравнительный анализ следующих алгоритмов сортировки:
\begin{itemize}
	\item блочная;
	\item бусинами;
	\item выбором;
\end{itemize}

В ходе выполнения лабораторной работы были решены все задачи:
\begin{itemize}
	\item разработаны требуемые алгоритмы;
	\item оценена трудоемкость рассматриваемых алгоритмов;
	\item проведен сравнительный анализ времени выполнения реализуемых алгоритмов и занимаемой памяти.
\end{itemize}

В результате исследования реализуемых алгоритмов по времени выполнения можно сделать следующие выводы:
\begin{enumerate}
	\item Алгоритм блочной сортировки демонстрирует значительно лучшую производительность на всех типах массивов, особенно выделяясь на неупорядоченных массивах (см. рисунки~\ref{img:noSort}~--~\ref{img:sort}). его реализация значительно превосходит другие алгоритмы, что делает его предпочтительным выбором для разнообразных данных.
	\item Алгоритм сортировки "Бусинами" показывает хорошие результаты на упорядоченных в обратном порядке и неупорядоченных массивах, но его  производительность снижается на массивах, упорядоченных по возрастанию (см. рисунки~\ref{img:noSort}~--~\ref{img:sort}). Это делает его подходящим для определенных случаев использования, особенно когда данные не предварительно отсортированы.
	\item Реализация алгоритма сортировки выбором  оказалась менее эффективной в сравнении с другими алгоритмами на всех типах массивов (см. рисунки~\ref{img:noSort}~--~\ref{img:sort}).
\end{enumerate}

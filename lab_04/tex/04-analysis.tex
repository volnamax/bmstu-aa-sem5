\chapter{Аналитический раздел}

В данном разделе будут рассмотрены три алгоритма сортировок: блочная сортировка, сортировка выбором
и сортировка бусинами.

\section{Алгоритм блочной сортировки}

Идея блочной сортировки заключается в разделении входной последовательности данных на несколько "блоков" одинакового размера, сортировке каждого блока, а затем объединении отсортированных блоков в одну отсортированную последовательность~\cite{sort_report}.

Алгоритм состоит из следующих шагов~\cite{sort_report}.

\begin{enumerate}
	\item \textit{Разделение массива.} Исходный массив разделяется на две равные (или почти равные) части. Это может быть сделано путем выбора опорного элемента (например, среднего элемента массива) и разделения массива на элементы, которые меньше опорного, и элементы, которые больше опорного.
	
	\item \textit{Сортировка каждой части.} Каждая из получившихся частей массива сортируется отдельно с использованием того же алгоритма блочной сортировки.
	
	\item \textit{Слияние отсортированных частей.} После того как обе части массива стали отсортированными, они объединяются в один отсортированный массив. Это делается путем сравнения элементов из обеих частей и добавления их в правильном порядке в результирующий массив. Этот процесс продолжается, пока не будут обработаны все элементы из обоих частей.
	
\end{enumerate}

\section{Алгоритм сортировки бусинами}

Алгоритм сортировки бусинами, известный также как гравитационная сортировка, представляет собой метод, использующий физические свойства, а именно гравитацию, для упорядочивания списка положительных целых чисел~\cite{article_bead}.

Принцип сортировки бусинами основывается на эффекте гравитационного падения бусин.
Визуализация процесса предполагает использование устройства, подобного счету, где каждый горизонтальный ряд бусин соответствует отдельному числу, а каждая бусина символизирует одну единицу этого числа~\cite{article_bead}.

Алгоритм состоит из следующих шагов~\cite{article_bead}.

\begin{enumerate}
	\item \textit{Инициализация.} Для начала, каждое число в массиве представляется в виде горизонтального ряда бусин, где каждая бусина эквивалентна одной единице числа.
	\item  \textit{Сортировка.} Под воздействием гравитации бусины падают, при этом бусины, представляющие более большие числа, скапливаются над бусинами меньших чисел. Процесс приводит к "смещению" бусин влево, формируя отсортированные столбцы. 
	\item \textit{Результат.} По завершении процесса подсчитывается количество бусин в каждом столбце, что и представляет собой отсортированный массив чисел.
\end{enumerate}

Алгоритм применим к натуральным числам. Можно сортировать и целые, но отрицательные числа придется обрабатывать отдельно от положительных.

\section{Алгоритм сортировки выбором}

Алгоритм сортировки выбором основан на поиске минимального (или максимального) значения на необработанном срезе массива или списка и последующем обмене этого значения с первым элементом необработанного среза. После каждого такого обмена размер необработанного среза уменьшается на один элемент~\cite{knut}.


Алгоритм состоит из следующих шагов~\cite{knut}.

\begin{enumerate}
	\item Находим минимальный (или максимальный) элемент в текущем массиве.
	\item Производим обмен этого элемента со значением первой неотсортированной позиции. Обмен не требуется, если минимальный (или максимальный) элемент уже находится на данной позиции.
	\item Сортируем оставшуюся часть массива, исключая из рассмотрения уже отсортированные элементы.
\end{enumerate}

